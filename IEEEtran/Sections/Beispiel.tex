\section{Beispielanwendung}
Um Cloud Computing an einem praktikablen Beispiel zu veranschaulichen, wurde eine Javascript-Anwendung für Nodejs entwickelt. In der Anwendung lassen sich Todos erstellen und nach erfolgreicher Abarbeitung löschen. Um die Anwendung in die Cloud auszulagern, wird das IaaS Angebot EC2 von Amazon AWS genutzt. Hier kann man virtuelle Server, sogenannte \glqq Instances\grqq{}, erstellen und dort die Anwendung deployen. Im Beispiel wurden zwei \glqq Instances\grqq{} verwendet und ein Load Balancer davor geschaltet. Der Load Balancer verteilt den eingehenden Anwendungs- und Netzwerkdatenverkehr auf mehrere Ziele, dadurch erhöht sich sowohl die Verfügbarkeit und die Fehlertoleranz. Die \glqq Instances\grqq{} können einer Auto Scaling-Gruppe zugewiesen werden, dadurch kann erreicht werden, dass immer ein Mindestanzahl von intakten \glqq Instances\grqq{} in Betrieb ist. Weitergehend kann auch die Anzahl dieser automatisch bei niedriger Last reduziert oder bei hoher Last erhöht werden. Die Ressourcennutzung des Load Balancers und der einzelnen \glqq Instances\grqq{} kann in der \glqq AWS Console\grqq{} im Menü unter Monitoring der jeweiligen Komponente betrachtet werden. Hier stehen unter anderem CPU-Nutzung, Netzwerk-auslastung, Speicher-operationen und auch Requests. Die erstellte Anwendung ist für geraume Zeit  \href{http://test-lb-1608271566.us-east-1.elb.amazonaws.com:8080/}{hier} verfügbar. 