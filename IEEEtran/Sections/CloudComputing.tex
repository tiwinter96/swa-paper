\section{Cloud Computing}
Seit der Entstehung des Begriffs \glqq Cloud Computing\grqq{} haben IT Firmen oder öffentliche Institute unterschiedliche Definitionen für den Begriff formuliert. Der Grundgedanke von Cloud Computing ist es jedoch Ressourcen oder Dienste über ein Netzwerk bereitzustellen oder zu nutzen. Es kann folgendermaßen definiert werden: Cloud Computing ist die Bereitstellung von Rechenressourcen über das Internet. Die Cloud Dienste ermöglichen sowohl Privatpersonen als auch Unternehmen Software oder Hardware zu nutzen, welche von einer außenstehenden Partei an einem anderen Standort zur Verfügung steht\cite{canada}. Cloud Computing ist eine spezifische Methode um skalierbare IT-bezogene Funktionen bei Bedarf über das Internet für mehrere Benutzer gleichzeitig nutzbar zu machen\cite{gartner}.
Offensichtlich gibt es keine einheitliche Definition für den Begriff \glqq Cloud Computing\grqq, jedoch hat sich die Definiton des amerikanischen National Institute of Standards and Technology (NIST) als weitläufig anerkannt herausgestellt. Das NIST bezeichnet Cloud Computing demnach als ein Modell, das einen einfachen und bedarfsgesteuerten Zugriff über ein Netzwerk zu einem geteilten Pool aus konfigurierbaren Rechenressourcen (bspw. Server, Speicherplatz, Anwendungen) ermöglicht. Diese Ressourcen sollen mit minimalen Verwaltungsaufwand oder durch den Dienst Provider bereitgestellt werden können. Cloud Computing setzt sich aus fünf Eigenschaften, drei Dienstmodellen und vier Einsatzmodellen zusammen\cite{nist_definition}.

\subsection{Eigenschaften}

\textbf{Selbstbedienung auf Nachfrage} 
Ein Verbraucher kann sich selbst Rechenressourcen bereitstellen ohne mit einem Mitarbeiter des Anbieters kommunizieren zu müssen\cite{nist_definition}.

\textbf{Breiter Netzwerkzugang}
Ressourcen sind über das Netzwerk verfügbar und können über Standardmechanismen von heterogenen Clients genutzt werden\cite{nist_definition}.

\textbf{Ressourcen Vereinigung}
Die Rechenressourcen des Anbieters sind gebündelt, um mehrere Verbraucher zu bedienen und ihnen dynamisch physikalische oder virtuelle Ressourcen auf Nachfrage zuzuweisen. Dafür wird ein Multi-Tenant-Modell benutzt. Diese Ressourcen sind beispielsweise Speicherplatz, Rechenleistung oder Netzwerkbandbreite. Der Verbraucher hat in der Regel keine Kenntnis darüber, wo sich die bereitgestellten Ressourcen befinden.\cite{nist_definition}.

\textbf{Schnelle Elastizität}
Die Ressourcen können dehnbar freigegeben und bereitgestellt werden, teilweise automatisch, um entsprechend der Nachfrage skalieren zu können\cite{nist_definition}.

\textbf{Messbarer Service}
Die Cloud-Systeme steuern automatisch die Ressourcennutzung durch Verwendung einer Messkapazität. Je nach Dienst bietet sich hierfür unterschiedliche Werte an, dies kann beispielsweise der Speicher oder aktive Benutzerkonten sein. Die Ressourcennutzung kann überwacht, kontrolliert und protokolliert werden, somit kann sowohl für den Verbraucher als auch für den Anbieter Transparenz für die Nutzung des benutzten Dienstes geschaffen werden\cite{nist_definition}.


\subsection{Dienst Modelle}
\subsubsection{Infrastructure as a Service (IaaS)} \label{IaaS}
Der Kunde bekommt vom Anbieter die Infrastruktur bereitgestellt um Operationen durchzuführen, dazu gehören Speicher, Hardware, sowie Netzwerk. Die Infrastruktur kann der Verbaucher benutzen um willkürliche Software darauf zu verwenden, einschließlich Betriebssysteme und Anwendungen. Der Verbraucher hat keine Kontrolle über die darunterliegende Infrastruktur, sprich Netzwerk und Hardware, aber er hat die volle Kontrolle über das Betriebssystem, Anwendungen, Speicher und möglicherweise eingeschränkten Zugriff auf die Netzwerkkomponenten beispielsweise Firewall-einstellungen\cite{nist_definition}.

\subsubsection{Platform as a Service (PaaS)} \label{PaaS} 
Der Kunde hat hier die Möglichkeit, eigens entwicklte oder erworbene Anwendungen auf der Infrastruktur des Anbieters bereit zu stellen. Der Verbaucher hat keine Kontrolle über die darunterliegende Infrastruktur, sprich Netzwerk, Server, Betriebssysteme, Speicherplatz. Aber er hat die Kontrolle über individuelle Anwendungseinstellungen sowie anwenderspezifische Einstellungen\cite{nist_definition}.

\subsubsection{Software as a Service (SaaS)} \label{SaaS}
Der Kunde hat hier die Möglichkeit, die bereitgestellten laufenden Anwendungen der Infrastruktur des Anbieters zu nutzen. Diese Anwendungen sind durch Client-Anwendungen über eine Schnittstelle erreichbar. Der Verbaucher hat keine Kontrolle über die darunterliegende Infrastruktur, sprich Netzwerk, Server, Betriebssysteme, Speicherplatz oder individuelle Anwendungseinstellungen mit Ausnahme eingeschränkter anwenderspezifischer Einstellungen\cite{nist_definition}.

\subsection{Kategorien}
\subsubsection{Private Cloud}
Diese Art von Cloud-Infrastruktur wird ausschließlich für eine einzige Organisation und ihren Mitarbeitern bereitgestellt. Die Infrastruktur kann von der Organisation selbst betrieben und verwaltet werden, von einer dritten Partei oder eine Kombination der beiden. Dabei kann die Infrastruktur innerhalb oder außerhalb der Räumlichkeiten dieser Organisation liegen\cite{nist_definition}.

\subsubsection{Community Cloud}
Diese Art von Cloud-Infrastruktur wird ausschließlich für eine spezielle Gemeinschaft an Verbrauchern mit einem gemeinsamen Interesse (bspw. Sicherheitsanforderungen oder Richtlinien) bereitgestellt. Die Infrastruktur kann von einer oder mehr Organisationen aus der Gemeinschaft selbst betrieben und verwaltet werden, von einer dritten Partei oder eine Kombination der beiden. Dabei kann die Infrastruktur innerhalb oder außerhalb der Räumlichkeiten dieser Organisationen liegen\cite{nist_definition}.

\subsubsection{Public Cloud}
Diese Art von Cloud-Infrastruktur wird für die Öffentlichkeit bereitgestellt. Die Infrastruktur wird von einem Unternehmen oder staatlichen Einrichtung selbst betrieben und verwaltet. Die Infrastruktur befindet sich in den Räumlichkeiten des Cloud-Providers\cite{nist_definition}.

\subsubsection{Hybrid Cloud}
Diese Art von Cloud-Infrastruktur ist ein Verbund aus zwei oder mehreren Cloud-Infrastrukturen(private, community oder public), welche weiterhin einzeln bestehen, aber mithilfe von standardisierten oder propietären Technologien die Portabilität von Daten und Anwendungen ermöglichen\cite{nist_definition}. Ein Teil der Service-Infrastruktur wird in der private Cloud bewältigt und ein anderer Teil in der Public Cloud. Das bietet eine bessere Kontrolle über die Daten und dennoch eine Möglichkeit Ressourcen nach Bedarf zu nutzen\cite{study_cc_cdb}.

\subsection{Besonderheiten}
\subsubsection{Konsistenz}
Wie jedes verteilte System haben auch Cloud-Architekturen Herausforderungen im Hinblick auf Konsistenz, Verfügbarkeit und Ausfallsicherheit zu meistern.
Das CAP-Theorem\cite{CAP} von Eric Brewer besagt, dass jedes System mit im Netzwerk verteilten Daten höchstens zwei von den drei gewünschten Eigenschaften haben kann:
\begin{itemize}
	\item \textit{Konsistenz} (\textbf{C}onsistensy)
	\item hohe \textit{Verfügbarkeit} (\textbf{A}vailability)
	\item \textit{Toleranz} bezüglich der Netzwerkpartitionen (\textbf{P}artition Tolerance)
\end{itemize}

Cloud-Speicherdienste, die zu den typischen Cloud-Diensten gehören, speichern ihre Daten redundant auf geographisch verteilten Servern, um somit einen rund um die Uhr Zugriff auf die Daten zu gewährleisten.
Bei redundanten Speichertechniken in Clouds, insbesonderen wenn die Skalierung eine weltweite Größe erreicht, wird es sehr schwierig eine strenge Konsistenz zu erreichen. Viele Anbieter der Cloud Dienste z.B. Amazon S3 können nur eine schwache Konsistenz
% , wie die \textit{kausale Konsistenz}, 
sicherstellen. Solche schwachen Konsistenzen erlauben eine bessere Performanz und hohe Verfügbarkeit, dafür kann es passieren, dass die Benutzer für eine bestimmte Zeit noch die alten Daten lesen.\cite{consistency-as-a-service} 

\subsubsection{Sicherheit}
Der Aspekt der Sicherheit gehört ebenso zu den Besonderheiten im Cloud Computing. So werden in \cite{seven-seq-risks} sieben spezielle Sicherheitsproblemme genannt, die von Kunden bzw. von Unternehmen bei der Wahl eines Cloud Anbieters beachtet werden sollen:

\textbf{1. Priviligierter Benutzerzugang.} Verarbeitung von sensitiven Daten außerhalb der Unternehmen bringt ein Risiko und die Benutzer sollten sich darüber im Klaren sein.

\textbf{2. Einhaltung gesetzlicher Vorschriften.} Kunden sind verantwortlich für die Sicherheit und Integrität ihrer Daten, selbst wenn diese in der Cloud liegen.

\textbf{3. Datenstandort.} Kunden sollen die Anbieter fragen, ob diese sich dazu verpflichten, Daten in bestimmten Rechtsordnungen zu speichern und zu verarbeiten, und ob diese eine vertragliche Verpflichtung eingehen, die lokalen Datenschutzbestimmungen für ihre Kunden einzuhalten.

\textbf{4. Datentrennung.} Kundendaten sollten von Daten anderer Kunden getrennt sein. Provider sollen nachweisen können, dass die Verschlüsselungsschemas von Experten entworfen wurden.

\textbf{5. Datenwiederherstellung.} Provider sollen Auskunft geben was mit den Daten im Falle einer Katastrophe passiert.

\textbf{6. Ermittlungssunterstützung.} Die Untersuchung von unangemessenen oder illegalen Aktivitäten sollte von Cloud Anbietern unterstütz sein. 

\textbf{7. Langfristige Rentabilität.} Daten sollten nicht verloren gehen, selbst wenn der Cloud Anbieter bankrott oder von einem anderen Unternehmen erworben wird.

% Diese drei Eigenschaften sind auch für das Cloud Computing von großer Bedeutung, da man hier mit einer horizontalen Skalierung zu tun hat.
% Bei Ausfall eines Knotens, \hl{z.B. einer Instanz der Webanwendung}, soll sichergestellt werden, dass die Anwendung weiterhin verfügbar ist.

% \hl{Sicherheit: 360 comparison. Seite 9, und SLA-Perspective paper ()}