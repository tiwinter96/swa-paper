\section{Einsatzgebiete}
Inzwischen ist die Bedeutung und Wert von IT in vielen Bereichen gestiegen und immer mehr Menschen möchten jederzeit und überall an Informationen kommen. Gebiete in denen Cloud Computing dafür genutzt wird, werden im folgenden Abschnitt vorgestellt.

\subsection{Gesundheitssystem}
Das Gesundheitssystem ist der Bereich in dem man wahrscheinlich den größten Nutzen davon ziehen kann. Die Cloud Computing Technologie kann das Gesundheitswesen um einiges effektiver und kostengünstiger machen. Die Patientendaten können in einer sogenannten elektronischen Gesundheitsakte EHR \glqq electronic health records\grqq{} gespeichert werden\cite{e-Health}. Hier sind die Medikamente, Krankheitsgeschichte, Röntgenbilder usw. dokumentiert. Diese EHR eines Patienten können für die Ärzte, Ernährungsberater oder Familienangehörige mit zugeschnittenen Berechtigungen entsprechend der benötigten Informationen zugänglich gemacht werden durch die Cloud Computing Technologie\cite{health}. Dadurch können die Informationen jederzeit abgerufen werden und müssen nicht erst beim entsprechenden Arzt angefordert werden. Dies kann mithilfe einer Cloud Computing Struktur realisiert werden:
\begin{itemize}
	\item SaaS für den Einsatz der Gesundheitssoftware, GUI und Authentifizierung
	\item PaaS Anwendungen, Datenbanken, Updates
	\item IaaS Physikalische Ressourcen zur Verarbeitung und Speicherung
\end{itemize} 

Dadurch können die Kosten für die Patienten reduziert, ärztliche Beratung über die entfernten Dienste angeboten und armen Gebieten kostengünstig eine Infrastruktur zur Verwaltung der Patientendaten angeboten werden\cite{recenttrends}.

\subsection{Bildungswesen}
Im traditionellen elektronischen Bildungsumfeld, können die Plattform und die Anwendungen innerhalb, außerhalb des Unternehmens oder eine Kombination aus diesen sein, abhängig von den Anforderungen der akademischen Organisation\cite{teaching}. Cloud Computing ermöglicht neue verbesserte Bildungsmöglichkeiten im Sinne von Online-Kursen oder Vorlesungsaufzeichnungen. Die benötigten Rechenressourcen und Software kann mithilfe der Cloud Computing Technologie problemlos bereit gestellt werden. Dadurch können die Effizienz und Kosten verbessert werden. Weiterhin ist das elektronische Lehren weitaus attraktiver als das traditionelle lernen mit Dokumenten\cite{teaching}.
Abgesehen der allgemeinen Vorteile von Cloud Computing wird den Bildungsinstitutionen mehr Freiheit zum Anpassen ihrer Produkte angeboten. Darüber hinaus bietet Cloud Computing eine offene Umgebung mit Zugang zu hochqualitativer Bildung für alle Dozenten, Studenten, wissenschaftliche Mitarbeiter und Forscher. Jede Person kann sich mit privaten Endgeräten überall und jederzeit Zugang verschaffen. Dozenten haben Zugriff auf ihre Lehrmaterialien und Programme remote und müssen diese nicht auf dem Laptop installiert haben. Dadurch, dass nicht einzelne Endgeräte erst authorisiert werden müssen oder Programmkompatibilitätsprobleme auftreten können, kann ein großer Verwaltungsaufwand erspart werden. Cloud Provider bieten die Möglichkeit die traditionellen Maschinen der Bildungsinstitutionen zu ersetzen. Dieses Angebot wird auch immer mehr von Institutionen genutzt. Anstatt, dass einzelne Universitäten sich mit Private Clouds bedienen, wäre auch eine gute Möglichkeit aus diesen Private Clouds gleichzeitig eine Hybrid Cloud mit anderen Universitäten aufzubauen. In dieser Hybrid Cloud könnten dann bestimmte Ressourcen aller teilnehmenden Universitäten geteilt werden\cite{recenttrends}.
