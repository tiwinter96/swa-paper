% For peer review papers, you can put extra information on the cover
% page as needed:
% \ifCLASSOPTIONpeerreview
% \begin{center} \bfseries EDICS Category: 3-BBND \end{center}
% \fi
%
% For peerreview papers, this IEEEtran command inserts a page break and
% creates the second title. It will be ignored for other modes.
\IEEEpeerreviewmaketitle

\section{Einleitung}
Cloud Computing ist eine IT-Infrastruktur, die in den letzten Jahren massiv an Bedeutung gewonnen hat.
Das Ziel von Cloud Computing ist es, über das Internet Rechenressourcen im Sinne von Speicherplatz, Rechenleistung oder Anwendungssoftware zur Verfügung zu stellen.
Zur Nutzung dieser Ressourcen werden technische Schnittstellen und Protokolle seitens Cloud-Provider angeboten. 
% Dabei haben sich verschiedene Servicemodelle entwickelt.
Immer mehr Unternehmen integrieren Cloud Lösungen in ihre IT-Infrastruktur, um zum einen die Kosten zu reduzieren und zum anderen dadurch an Flexibilität zu gewinnen. 
Zu den größten Cloud-Providern gehören unter anderem AWS, Microsoft und Google.
In folgenden Kapiteln wird ein Überblick über die Definitionen, Kategorien und Einsatzgebiete von Cloud Computing gegeben.
Es werden auch die Besonderheiten der Cloud Computing Architekturen erklärt.
Anschließend findet eine Erläuterung der Cloud Computing Architekturen statt, sowohl aus der Sicht eines Providers als auch aus der Sicht einer Anwendung.
Da die Begriffe Cloud Computing und Grid Computing oft miteinander verwechselt werden, wird hier versucht eine Grenze zwischen den beiden Technologien zu finden.
