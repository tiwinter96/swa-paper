\section{Abgrenzung zu Grid Computing}
Der Ansatz -- die Infrastruktur, Rechenleistung sowie Anwendungen oder Speicherplatz über das Internet für verschiedene Zwecke zur Verfügung zu stellen -- ist kein neuer.
Diese Idee, oder zumindest eine ähnliche, entstand schon vor mehreren Jahren mit dem Begriff \textbf{Grid Computing}.
Im Folgenden werden die wesentlichen Unterschiede zwischen dem Cloud Computing und dem Grid Computing erläutert.
 
\subsection{Grid Computing}
In \cite{grid-checklist} schlägt der Autor eine 3-Punkte-Checkliste vor, anhand deren sich ein Grid-System \hl{identifizieren} lässt. Demnach ist Grid ein System, das:
\begin{enumerate}
  \item Ressourcen koordiniert, die keiner zentralisierten Kontrolle unterliegen
  \item dabei offene und allgemeine Standardprotokolle und Schnittstellen verwendet
  \item um nicht-triviale \glqq Quality of Services\grqq{} zu liefern
\end{enumerate}

Cloud Computing überschneidet sich nicht nur mit Grid Computing, es ist in der Tat aus Grid Computing entstanden und bildet dessen Rückgrad auf dem seine Infrastruktur aufbaut\cite{360-degree-compared}.

Das wirkliche und spezielle Problem, das dem Konzept des Grids zugrunde liegt,
ist die koordinierte gemeinsame Nutzung von Ressourcen und das Lösen von Problemen
in einer dynamischen institutionsübergreifenden virtuellen Organisation.
Dabei ist mit \glqq gemeinsamer Nutzung\grqq{} der direkte Zugang zu Computern, Software, Daten und anderen Ressourcen,
die für das Lösen von Problemen in industriellen, wissenschaftlichen und technischen Bereichen benötigt werden, zu verstehen.
Das Teilen der Ressourcen beinhaltet klar definierte Regeln, die notwendigerweise stark kontrolliert werden.
Was geteilt werden kann, wer teilen darf und unter welchen Bedingungen geteilt werden darf ist dabei klar und sorgfältig definiert.
Eine Gruppe von Individuen und/oder Institutionen, die durch solche Regeln definiert sind,
bilden die sogenannten \textbf{Virtuellen Organisationen}.\cite{anatomy-of-grid}

\hlred{Die Ressourcen einer virtuellen Organisation können geographisch verteilt sein.}\todo{Zitat finden}

\subsection{Anwendungsbereich}

\subsection{Geschäftsmodell}

\subsection{Architektur}

\subsection{Ressourcenverwaltung}


\subsection{Fazit}
\hlred{Während Cloud Computing hauptsächlich Lösungen für verschiedene webbasierte Businessfälle bietet, wird Grid Computing dagegen mehr in wissenschaftlichen Projekten bei rechenintensiven Aufgaben eingesetzt.
Grid Computing systems are not intended to be used in business purposes. e.i. there is no business model behind it. The users or organisations who want to make use of Grid Computing simply join a VO and start consuming
Aus Benutzersicht werden Ressourcen bei Cloud Computing zentral verwaltet.
Der Benutzer kann, im Web, vom Cloud Provider zur Verfügung gestellte Schnittstelle verwenden,
um Ressourcen bei Bedarf hinzuzufügen oder zu entfernen.
Grid Computing ist in dieser Hinsicht dezentralisiert. 
Die Ressourcen können sich in unterschiedlichen virtuellen Organisationen befinden.} 

[Wie sieht es für die Zukunft aus?]

\hl{Wird man in der Zukunft komplett auf Grid Computing verzichten können und stattdessen auf Cloud Computing Lösungen zugreifen?}
